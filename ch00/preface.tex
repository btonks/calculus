\formatlikesection{Preface}

Calculus isn't a hard subject.

Algebra is hard. I still remember my encounter with
algebra. It was my first taste of abstraction in mathematics,
and it gave me quite a few black eyes and bloody noses.

Geometry is hard. For most people, geometry is the first time
they have to do proofs using formal, axiomatic reasoning.

I teach physics for a living. Physics is hard. There's a reason
that people believed Aristotle's bogus version of physics for
centuries: it's because the real laws of physics are counterintuitive.

Calculus, on the other hand, is a very straightforward subject that
rewards intuition, and can be easily visualized. Silvanus Thompson,
author of one of the most popular calculus texts ever written,
opined that ``considering how many fools can calculate, it is
surprising that it should be thought either a difficult or
a tedious task for any other fool to master the same tricks.''

Since I don't teach calculus, I can't require anyone to read this
book. For that reason, I've written it so that you can go through it
and get to the dessert course without having to eat too many Brussels
sprouts and Lima beans along the way. The development of any mathematical
subject involves a large number of boring details that have little
to do with the main thrust of the topic. These details I've relegated
to a chapter in the back of the book, and the reader who has an
interest in mathematics as a career --- or who enjoys a nice heavy
pot roast before moving on to dessert --- will want to read those
details when the main text suggests the possibility of a detour.

